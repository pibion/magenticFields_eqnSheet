\documentclass[10pt,landscape]{article}
\usepackage{multicol}
\usepackage{calc}
\usepackage{ifthen}
\usepackage[landscape]{geometry}
\usepackage{amsmath,amsthm,amsfonts,amssymb}
\usepackage{color,graphicx,overpic}
\usepackage{hyperref}
\usepackage{nonfloat}
\usepackage{array}
\usepackage{wrapfig}
\newcolumntype{L}[1]{>{\raggedright\let\newline\\\arraybackslash\hspace{0pt}}m{#1}}
\newcolumntype{C}[1]{>{\centering\let\newline\\\arraybackslash\hspace{0pt}}m{#1}}
\newcolumntype{R}[1]{>{\raggedleft\let\newline\\\arraybackslash\hspace{0pt}}m{#1}}

\newenvironment{Figure}
  {\par\medskip\noindent\minipage{\linewidth}}
  {\endminipage\par\medskip}

\pdfinfo{
  /Title (example.pdf)
  /Creator (TeX)
  /Producer (pdfTeX 1.40.0)
  /Author (PHYS 2331-002)
  /Subject (Electrostatics)
  /Keywords (pdflatex, latex,pdftex,tex)}

\title{Circuits Exam Cheat Sheet}

% This sets page margins to .5 inch if using letter paper, and to 1cm
% if using A4 paper. (This probably isn't strictly necessary.)
% If using another size paper, use default 1cm margins.
\ifthenelse{\lengthtest { \paperwidth = 11in}}
    { \geometry{top=.5in,left=.5in,right=.5in,bottom=.5in} }
    {\ifthenelse{ \lengthtest{ \paperwidth = 297mm}}
        {\geometry{top=1cm,left=1cm,right=1cm,bottom=1cm} }
        {\geometry{top=1cm,left=1cm,right=1cm,bottom=1cm} }
    }

% Turn off header and footer
\pagestyle{empty}

% Redefine section commands to use less space
\makeatletter
\renewcommand{\section}{\@startsection{section}{1}{0mm}%
                                {-1ex plus -.5ex minus -.2ex}%
                                {0.5ex plus .2ex}%x
                                {\normalfont\large\bfseries}}
\renewcommand{\subsection}{\@startsection{subsection}{2}{0mm}%
                                {-1explus -.5ex minus -.2ex}%
                                {0.5ex plus .2ex}%
                                {\normalfont\normalsize\bfseries}}
\renewcommand{\subsubsection}{\@startsection{subsubsection}{3}{0mm}%
                                {-1ex plus -.5ex minus -.2ex}%
                                {1ex plus .2ex}%
                                {\normalfont\small\bfseries}}
\makeatother

% Define BibTeX command
\def\BibTeX{{\rm B\kern-.05em{\sc i\kern-.025em b}\kern-.08em
    T\kern-.1667em\lower.7ex\hbox{E}\kern-.125emX}}

% Don't print section numbers
\setcounter{secnumdepth}{0}


\setlength{\parindent}{0pt}
\setlength{\parskip}{0pt plus 0.5ex}

%My Environments
\newtheorem{example}[section]{Example}
% -----------------------------------------------------------------------

\begin{document}
\raggedright
\footnotesize
\begin{multicols}{3}


% multicol parameters
% These lengths are set only within the two main columns
%\setlength{\columnseprule}{0.25pt}
\setlength{\premulticols}{1pt}
\setlength{\postmulticols}{1pt}
\setlength{\multicolsep}{1pt}
\setlength{\columnsep}{2pt}

\begin{center}
     \Large{\underline{Circuits Exam Cheat Sheet}} \\
\end{center}

\section{Definitions and Units}
%\begin{center}
    \begin{tabular}{| L{1.7cm} | L{1.5cm} | L{2cm} | R{1cm} |}
    \hline
    Name & Scaler or Vector? & Symbol & Units \\ \hline
    ``Force" & vector & $\vec{F}$ & N \\ \hline
    ``Work" & scaler & $W \equiv \int_{path}\vec{F}\cdot\vec{dl}$ & J \\ \hline
    ``Potential Energy" & scaler & $\Delta U\equiv -W$ & J \\ \hline
    ``Electric Field" & vector & $\vec{E}\equiv\vec{F}/q$ & N/C \\ \hline
    ``Electric Potential", or "Voltage" & scaler & $\Delta V\equiv\Delta U/q$ & J/C $\equiv$ V\\
    \hline
    ``Current'' & scaler & $I = \frac{dQ}{dt}$ & C/s \\\hline
    ``Resistance'' & scaler & $R\equiv V/I$ & $Ohms, \Omega$ \\ \hline
    ``Capacitance'' & scaler & $C \equiv |Q/V|$ & Farads, F \\\hline
    ``Power'' & scaler & $P \equiv \frac{dW}{dt}$ & $\frac{J}{s} \equiv W (Watts)$ \\ \hline
    \end{tabular}
%\end{center}

\section{Batteries}
A battery is a chemical machine that provides a constant potential (voltage) across its terminals.  

\begin{Figure}
\centering
\includegraphics[height=3cm]{nonideal_batter_matterAndInteractions.PNG}
%\caption{text}
\end{Figure}	

\section{Conservation of Charge and Energy (Kirchoff's Laws)}
In any circuit loop, the total potential difference must be zero, or $\sum \Delta V = 0$

At any node in a circuit, the current flowing in must be the same as the current flowing out, or $i_1 = i_2 + i_3$.

\begin{Figure}
\centering
\includegraphics[height=3cm]{kirchoff_node_matterAndInteractions.PNG}
%\caption{text}
\end{Figure}

\section{Circuit Elements in Parallel and Series}
Resistors in series: $R = R_1 + R_2 + R_3 + ...$

Resistors in parallel: $\frac{1}{R} = \frac{1}{R_1} + \frac{1}{R_2} + ...$

Capacitors in series: $\frac{1}{C} = \frac{1}{C_1} + \frac{1}{C_2} + ...$

Capacitors in parallel: $C = C_1 + C_2 + C_3 + ...$

\section{Power dissipation in circuits}
Electrons flowing through resistive material warm up the material.

The power dissipated in a circuit can be written $P = IV = I^2R$.

\section{Charging and Discharging a Capacitor}
Capacitors store charge - and how fast they discharge is determined by both the resistance and capacitance of the circuit.
The charge on a discharging capacitor decays exponentially: $Q(t) = Q_0 e^{-t/RC}$.

\section{Electric Field}
Electric fields are relevant to circuits when working with capacitors (to calculate $\Delta V = -\int\vec{E}\cdot\vec{dl}$ one needs to know the electric field $\vec{E}$.  In addition, when considering the operation of circuits one needs to know the electric field to evaluate the forces experienced by the mobile electrons.

\subsection{Electric Field due to a point-like Object}
The electric field due to a small, point-like object with net charge $Q$ is measured to be $\vec{E}=\frac{1}{4\pi\epsilon_0}\frac{Q}{r^2}\hat{r}$.

All electric fields can be derived from this, together with the principle of superposition: $\vec{E} = \vec{E_1} + \vec{E_2} + \vec{E_3} + ...$ and allows us to calculated the electric field of an arbitrary, fixed charge distribution.

\subsection{Electric Field due to a charged wire}
The magnitude of the electric field due to a wire with length L and net charge Q is approximately $\frac{1}{4\pi\epsilon_0}\frac{2Q/L}{r}$, where $r$ is the smallest possible distance to the wire.  This is a good approximation if the length of the wire $L$ is much larger than the distance from the wire $r$.  The direction of the electric field is radially outward from the wire if the charge is positive and radially toward the wire if the charge is negative.

\begin{Figure}
\centering
\includegraphics[height=5cm,angle=270]{Efield_wire_matterAndInteractions.PNG}
%\caption{text}
\end{Figure}	


\subsection{Electric Field due to a charged plane}
The magnitude of the electric field due to a large, charged plane is approximately $\frac{Q/A}{2\epsilon_0}$.  (Note that this does not depend on the distance from the plate.)  The direction of the electric field is away (perpendicular) from the plane if the plane is positively charged and towards (perpendicular) the plane if the plane is negatively charged.
\begin{Figure}
\centering
\includegraphics[height=3cm]{eField_plane}
%\caption{text}
\end{Figure}

\section{Electric Force}
The electric field is defined to be the force per charge, $\vec{E} = \vec{F}/q$.

If we know the electric field, this allows us to easily calculate the force $\vec{F}$ felt by a point-like object with net charge $q$, $\vec{F} = q\vec{E}$.

\subsection{Electric Field inside a Dielectric (Insulating) Material}
A dielectric (or insulating) material will \emph{polarize} when exposed to an electric field - the electrons inside the dielectric will move as much as they are able in response to the electric field.

The effect is that the electric field inside the dielectric is reduced.

We describe the amount a dielectric is able to reduce the electric field by the ``dielectric constant'' $K$, where $|\vec{E}| = |\vec{E_0}|/K$.

\section{Parallel Plate Capacitor}
Capacitors consisting of parallel, conducting plates are common circuit elements.  Like all capacitors, their capacitance can be found through $C\equiv Q/V$, where $V$ is the potential difference between the conducting plates, $\Delta V = -\int\vec{E}\cdot\vec{dl}$.

By calculating $V$ from the electric field $\vec{E}$ between the plates, the capacitance of a parallel plate capacitor can be written $C = \epsilon_0\frac{A}{d}$, where $A$ is the area of one of the plates and $d$ is the distance between the plates.

If a dielectric (insulating) material with dielectric constant $K$ fills the gap between the conducting plates, this changes the electric field, which changes the voltage between the plates, which changes the capacitance: $C = \epsilon_0 K\frac{A}{d}$

\section{Work and Potential Energy}
The change in potential energy between the start and stop of a path is defined to be negative the work done on that path, or $\Delta U \equiv -W \equiv -\int\vec{F}\cdot\vec{dl}$.  Note that this definition holds no matter what causes the force - gravity or charge.

\section{Electric Potential}
When considering circuits We \emph{only} talk about the change in electric potential, $\Delta V \equiv \Delta U / q$.  Note that the definition of $\Delta U$ above means that $\Delta V \equiv -\int\vec{E}\cdot\vec{dl}$.

The integral definition of the electric potential can be turned into a differential definition: $\vec{E} = -\vec{\nabla}{V} = -\frac{\partial V}{\partial x}\hat{x} -\frac{\partial V}{\partial  y}\hat{y} - \frac{\partial V}{\partial z}\hat{z}$.

%In some cases we talk about the electric potential at a point in space.  By convention, we mean $V_A = V_A - V_{\inf} = V_A - 0$.

%The electric potential at a point a distance $r$ from a point charge $Q$ is $V = \frac{1}{4\pi\epsilon_0}\frac{Q}{r}$.

\section{Energy}
Any region with an electric field $\vec{E}$ is storing energy.  The energy density for this region is $u = \frac{1}{2}\epsilon_0|\vec{E}|^2$.

\section{Vectors}
If point $S$ is located at $(x_1, y_1, z_1)$ and point $P$ is located at $(x_2, y_2, z_2)$ then the vector from point $S$ to point $P$ can be written: $\vec{r} = (x_2 - x_1)\hat{x} + (y_2 - y_1)\hat{y} + (z_2 - z_1)\hat{z}$.  The length of that vector is also called its magnitude and is denoted $|\vec{r}|$, where $|\vec{r}| = \sqrt{(x_2 - x_1)^2 + (y_2 - y_1)^2 + (z_2 - z_1)^2}$.  The magnitude of $\vec{r}$ is sometimes denoted simply by $r$.

\subsection{Unit Vectors}
A unit vector is a vector with magnitude one (``unity'') and is denoted with a hat: $\hat{r} = \vec{r}/|\vec{r}|$.

\subsection{Dot Product}
The dot product between two vectors results in a scaler and measures the ``overlap'' between two vectors.  If two vectors are perpendicular, their dot product is zero.  

\begin{align*}
\vec{a}\cdot\vec{b} &= a_x*b_x + a_y*b_y + a_z*b_z \\
                    &= |\vec{a}||\vec{b}|\cos{\theta}
\end{align*}
                  
Note that $\hat{x}\cdot\hat{x} = 1$ while $\hat{x}\cdot\hat{y} = 0$.

\begin{Figure}
\centering
\includegraphics[width=0.9\textwidth]{r_vector.PNG}
%\caption{text}
\end{Figure}

\section{Constants}

$\frac{1}{4\pi\epsilon_0} = 9\times 10^9 N/Cm^2$ \newline
$\epsilon_0 = 8.85\times 10^{-12} {Cm^2/N}$ \newline
charge of an electron (denoted $-e$) = $-1.6\times 10^{-19} C$\newline
charge of a proton (denoted $e$) = $1.6\times 10^{-19} C$\newline
mass of an electron = $9.1\times 10^{-32} kg$ \newline
mass of a proton = $1.673\times 10^{-27} kg$ \newline
acceleration due to gravity (near the Earth's surface) = $9.8 N/kg = 9.8 m/s^2$ \newline
gravitational force constant $G$ = $6.67\times 10^{-11} Nm^2/kg$

% You can even have references
\rule{0.3\linewidth}{0.25pt}
\scriptsize
\bibliographystyle{abstract}
\bibliography{refFile}
\end{multicols}


\end{document}
